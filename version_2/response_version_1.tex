% Authors' response
\thispagestyle{empty}
\section*{Authors' response}
To acp-2017-34-referee-report-2:
\begin{itemize}
\item Minors:
  \begin{itemize}
  \item[]Figure 1: A more thorough explanation has been added to the corresponding caption. Concerning the $+$ sign: \emph{[...] Two examples of catalytic cycles of ozone depletion involving bromine. A $+$ sign indicates increasing order of catalytic complexity. Reactants are shown in red, catalysts in black, and products in blue. Photochemical reactions are indicated by a violet wave.}
  \item[]P8 L4: Changed 'hPa' to 'ppt'.
  \item[]P10 L11: The sentence states that VSLS are influenced by 'the speed-up BDC'. Added a reference accordingly~\citep{GRL:Hossaini2012}.
  \item[]P11 figure5: Changed the caption in accordance. Caption of Figure 9 has changed accordingly.
  \item[]Section 5: Some sentences have been rearranged and a discussion of results from~\citet{ACP:Yang2014} in concert to results of this study has been included: \emph{In more detail,\citet{ACP:Yang2014} investigated the influence of brominated VSLS in combination with differing levels of atmospheric chlorine load on ozone. In concert with our results, although by doubling the initial amount of VSLS on a varying bromine background from anthropogenic sources, they have found a significant decrease of ozone in the tropical UTLS and polar region in the order of 2--4}\% and slight, insignificant increases in the antarctic mid-stratosphere~\citep[Fig.~1e]{ACP:Yang2014}
  \end{itemize}
\item General comments:
  \begin{itemize}
  \item[]Section 5: A reference to~\citet{ACP:Braesicke2013} has been added considering dynamical feedbacks, through VSLS induced changes in ozone to BDC and ozone distribution.
  \end{itemize}
\end{itemize}
\newpage
